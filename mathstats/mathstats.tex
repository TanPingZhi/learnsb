\documentclass[10pt, landscape]{article}
\usepackage{amsmath}
\usepackage[scaled=0.92]{helvet}
\usepackage{calc}
\usepackage{multicol}
\usepackage[a4paper,margin=3mm,landscape]{geometry}
\usepackage{amsmath,amsthm,amsfonts,amssymb}
\usepackage{color,graphicx,overpic}
\usepackage{hyperref}
\usepackage{newtxtext} 
\usepackage{enumitem}
\usepackage[table]{xcolor}
\usepackage{mathtools}
\setlist{nosep}
% for including images
\graphicspath{ {./images/} }

\pdfinfo{
  /Title (springboot.pdf)
  /Creator (TeX)
  /Producer (pdfTeX 1.40.0)
  /Author (Tan Ping Zhi)
  /Subject (CSIT)
/Keywords (springboot, nus,cheatsheet,pdf)}

% Turn off header and footer
\pagestyle{empty}

% redefine section commands to use less space
\makeatletter
\renewcommand{\section}{\@startsection{section}{1}{0mm}%
  {-1ex plus -.5ex minus -.2ex}%
  {0.5ex plus .2ex}%x
{\normalfont\large\bfseries}}
\renewcommand{\subsection}{\@startsection{subsection}{2}{0mm}%
  {-1explus -.5ex minus -.2ex}%
  {0.5ex plus .2ex}%
{\normalfont\normalsize\bfseries}}
\renewcommand{\subsubsection}{\@startsection{subsubsection}{3}{0mm}%
  {-1ex plus -.5ex minus -.2ex}%
  {1ex plus .2ex}%
{\normalfont\small\bfseries}}%
\makeatother

\renewcommand{\familydefault}{\sfdefault}
\renewcommand\rmdefault{\sfdefault}
%  makes nested numbering (e.g. 1.1.1, 1.1.2, etc)
\renewcommand{\labelenumii}{\theenumii}
\renewcommand{\theenumii}{\theenumi.\arabic{enumii}.}
\renewcommand\labelitemii{•}
\renewcommand\labelitemiii{•}

\definecolor{mathblue}{cmyk}{1,.72,0,.38}
\everymath\expandafter{\the\everymath \color{mathblue}}

% Don't print section numbers
\setcounter{secnumdepth}{0}

\setlength{\parindent}{0pt}
\setlength{\parskip}{0pt plus 0.5ex}
%% adjust spacing for all itemize/enumerate
\setlength{\leftmargini}{0.5cm}
\setlength{\leftmarginii}{0.5cm}
\setlist[itemize,1]{leftmargin=2mm,labelindent=1mm,labelsep=1mm}
\setlist[itemize,2]{leftmargin=4mm,labelindent=1mm,labelsep=1mm}
\setlist[itemize,3]{leftmargin=4mm,labelindent=1mm,labelsep=1mm}

% adding my commands
\input{../commands/style-helpers.tex}
\input{../commands/code.tex}
\DeclareMathOperator{\pairwisemax}{pairwise-max}

\definecolor{myred}{rgb}{1,0,0} % RGB color model
\newcommand{\hlred}[1]{{\sethlcolor{myred}\hl{#1}}}

% -----------------------------------------------------------------------

\begin{document}
\raggedright
\footnotesize
\begin{multicols*}{2}
  % multicol parameters
  \setlength{\columnseprule}{0.25pt}

  \begin{center}
    \fbox{%
    \parbox{0.8\linewidth}{\centering \textcolor{black}{
      {\Large\textbf{Week 1}}
      \\ \normalsize{CSIT 2024}}
    \\ {\footnotesize \textcolor{gray}{github/TanPingZhi}}
    }%
    }
  \end{center}

  \section{1. Probabilty}
  \subsection{7. Prove that $P(A \cap B) \ge P(A) + P(B) - 1$}
  \begin{align}
    P(A \cup B) & = P(A) + P(B) - P(A \cap B) \ge 1 \\
    1           & \ge P(A) + P(B) - P(A \cap B)     \\
    P(A \cap B) & \ge P(A) + P(B) - 1
  \end{align}

  \subsection{8. Prove $P(\bigcup\limits_{i=1}^{n}A_i) \le  \sum \limits_{i=1}^{n} P(A_i)$}
  Through sheer force of will, \\
  Consider the marginal $B_j = A_j - \bigcup\limits_{i=1}^{i-1}A_i$
  \begin{align*}
    P(\bigcup\limits_{i=1}^{n}A_i) & = P(\bigcup\limits_{j=1}^{n}B_j)  \\
                                   & = \sum \limits_{j=1}^{n} P(B_j)   \\
                                   & \le \sum \limits_{i=1}^{n} P(A_i)
  \end{align*}

  \subsection{15. How many different meals can be made from 4 kinds of meat, 6 vegetables, and 3 starches if a meal consists of one selection from each group?}
  $4*6*3 = 72$

  \subsection{35. Prove the following 2 identities both algebraically and combinatorially}
  % \subsubsection{a. $\binom{n}{r}$}
  $\binom{n}{r} = \binom{n}{n-r}$
  \begin{align*}
    \binom{n}{r} & = \frac{n!}{r!(n-r)!} \\
                 & = \frac{n!}{(n-r)!r!} \\
                 & = \binom{n}{n-r}
  \end{align*}
  $\binom{n}{r} = \binom{n-1}{r-1} + \binom{n-1}{r}$ \\
  Consider an objeect that is either in the set or not in the set.
  If it is in the set, then there are $r-1$ objects left to choose from $n-1$ objects.
  If it is not in the set, then there are $r$ objects left to choose from $n-1$ objects.

  \section{2. Random Variables}
  \subsection{21. If X is a geometric RV, show that}
  $P(X > n + k - 1 | X > n - 1) = P(X > k)$
  \begin{align*}
    P(X=k)                       & = (1-p)^{k-1}p                                                                                     \\
    P(X > n + k - 1 | X > n - 1) & = \frac{\sum_{i = n + k}^{\infty} P(X = i)}{\sum_{i = n}^{\infty} P(X = i)}                        \\
                                 & = 1 - \frac{\sum_{i = n}^{n + k - 1} (1-p)^{i-1}p}{\sum_{i = n}^{\infty} (1-p)^{i-1}p}             \\
                                 & = 1 - \frac{\sum_{i = n}^{n + k - 1} (1-p)^{i-1}}{\sum_{i = n}^{\infty} (1-p)^{i-1}}               \\
                                 & = 1 - \frac{(1-p)^{n-1}\sum_{i = 0}^{k - 1} (1-p)^{i}}{(1-p)^{n-1}\sum_{i = 0}^{\infty} (1-p)^{i}} \\
                                 & = 1 - \frac{\sum_{i = 0}^{k - 1} (1-p)^{i}}{\sum_{i = 0}^{\infty} (1-p)^{i}}                       \\
                                 & = 1 - \frac{\sum_{i = 0}^{k - 1} (1-p)^{i}}{\frac{1}{p}}                                           \\
                                 & = 1 - \sum_{i = 0}^{k - 1} (1-p)^{i}p                                                              \\
                                 & = P(X > k)
  \end{align*}

  \subsection{In a sequence of independent trials with probability p of success, what is the
  probability that there are r successes before the kth failure?}
  \begin{align*}
    \text{Consider the combination of k-1 failures and r successes. The last trial wil be the kth failure.} \\
    (1-p) * \binom{r + k - 1}{k-1} * (1-p)^{k-1} * p^r &= (1-p)^k * p ^ r * \binom{r + k - 1}{k-1}
  \end{align*}



\end{multicols*}
\end{document}
